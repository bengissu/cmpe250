\documentclass[12pt]{article}

\usepackage[mathcal]{euscript}
\usepackage{mathrsfs}
\usepackage{amsfonts,amssymb,amsbsy,amsmath}
\usepackage{amsthm,verbatim}
%\usepackage[mathscr]{eucal}
\usepackage{latexsym}
\usepackage{yfonts}
\usepackage[latin5]{inputenc}

\allowdisplaybreaks
\parindent=0in
\parskip=10pt
    
\pagestyle{empty}
\textwidth 18.0cm
\topmargin -2.5cm
\textheight 27.0cm
\oddsidemargin -1.0cm
\parindent 0.0cm

\begin{document}
%\leftline{Student name:} 
%leftline{Student \#}

\begin{center}
{\bf Bo\~gazi\c{c}i University, Math 327 Number Theory},  C. Y. Y{\i}ld{\i}r{\i}m \\ 
\mbox{} \\
{\bf Examples on Chapter 2 of Baker's book}, \, November 2020 \\
\mbox{} \\
%{\bf Questions}
\end{center}


{\bf 1)} (Baker, 2.10 Ex.(i); NZM Problem 4.3.7) Evaluate $\displaystyle \sum_{d \mid n}\mu(d)\sigma(d)$.
\vskip .3cm
Since both of $\mu$ and $\sigma$ are multiplicative functions, by the result on p. 9 of Baker (which is Theorem 4.4 of NZM), the given sum is also a multiplicative function of $n$. Hence to find the values of this sum at all integers $n$, it is enough to find its values at the prime powers.
Now, for $n =p^j,\, j\in \mathbb{Z}^{+}$, we have, since $\mu(p^2) = \mu(p^3) = \cdots  = 0$, 
$\displaystyle \sum_{d \mid p^j}\mu(d)\sigma(d) = \mu(1)\sigma(1) + \mu(p)\sigma(p)= 1\cdot 1 + (-1)\cdot(p+1) = -p$. So, for $n=p_{1}^{j_1}\cdots p_{k}^{j_k},
\, ( j_1, \ldots, j_k > 0)$, we have $\displaystyle \sum_{d \mid n}\mu(d)\sigma(d) = (-1)^{k}p_1 \cdots p_k$.

{\bf 2)} (Baker, 2.10 Ex.(iii)) Prove that $\displaystyle {1\over n} \sum_{1\leq a \leq n \atop (a,n)=1}a = {\phi(n)\over 2}$ for $n >1$.
\vskip .3cm
$ \displaystyle \sum_{1\leq a \leq n \atop (a,n)=1}a = \sum_{1 \leq a \leq n}a\sum_{d\mid (a,n)}\mu(d)$ by the basic property of the M\"{o}bius function (Baker p.10;
NZM Theorem 4.7). Now, writing $a =db$, we have
\begin{align*}
\sum_{1 \leq a \leq n}a\sum_{d\mid (a,n)}\mu(d) & = \sum_{d\mid n}\mu(d)\sum_{1 \leq a \leq n \atop d \mid a}a =  \sum_{d\mid n}d\mu(d)\sum_{b \leq {n\over d}}b =  \sum_{d\mid n}d\mu(d) \big({1\over 2}{n\over d}({n\over d}+1)\big) \\
& = {n^2\over 2}\sum_{d \mid n}{\mu(d)\over d} + {n\over 2}\sum_{d \mid n}\mu(d) =  {n^2\over 2}\sum_{d \mid n}{\mu(d)\over d}  \; (\mathrm{for} \; n>1) \\
& =  {n \phi(n)\over 2} \; (\mathrm{ Baker \, p. 11; NZM \, formula \, (4.1) \, on \, p. 195}) 
\end{align*}
(Note that for $n=1$, the term $\displaystyle {n\over 2}\sum_{d \mid n}\mu(d)$ in the penultimate line of the arrayed equations above is not $0$, instead it is 
$\displaystyle {1\over 2}$). (Baker 2.10 Ex.(iv), and the result of NZM Problem 4.3.12 can be done similarly).
\vskip .3cm

{\bf 3)} (Baker 2.10 Ex.(xi)) Prove that $\displaystyle \sum_{n=1}^{\infty}{\phi(n)x^n\over 1-x^n} = {x\over (1-x)^2},\, (|x| < 1)$. (Series of this kind are called Lambert series).
\vskip .3cm
We need $|x|<1$ for convergence. We have, writing $m=nk$ and using Theorem 4.6 of NZM (or see Baker p. 10), 
\begin{align*}
\sum_{n=1}^{\infty}{\phi(n)x^n\over 1-x^n} = \sum_{n=1}^{\infty}\phi(n)\sum_{k=1}^{\infty}x^{nk} = \sum_{m=1}^{\infty}x^m \sum_{n\mid m}\phi(n)
= \sum_{m=1}^{\infty}m x^m = {x\over (1-x)^2}.
\end{align*}
\vskip .3cm

{\bf 4)} (Baker 2.10 Ex.(ii); NZM formula (8.42) The arithmetic function  $\Lambda(n)$, known as von Mangoldt's function, is defined as follows: 
\begin{align*}
\Lambda(n) = \left\{
\begin{array}{llll}
\log p & \mbox{if $n=p^{a}$ with $p$ prime and $a\in \mathbb{Z}^{+}$} \\
0 & \mbox{otherwise}. \end{array}
\right.
\end{align*}
Evaluate $\displaystyle \sum_{d\mid n}\Lambda(d)$. Express $\displaystyle \sum_{n=1}^{\infty}{\Lambda(n)\over n^s}$ in terms of $\zeta(s)$.
\vskip .3cm
Here we have to take $s >1$, or more generally for $s$ a complex number $\Re s >1$, in order to have convergence and absolute convergence in the
concerned infinite series and infinite products. Here we shall not be very much concerned with convergence issues, but instead we will be dealing with
manipulations on series and products.
The Riemann zeta-function, $\zeta(s)$ is defined as
\begin{align*}
\zeta(s) := \sum_{n=1}^{\infty}{1\over n^s},\quad (s>1 \, \mathrm{for}\, s\in \mathbb{R}; \Re s >1\, \mathrm{for}\, s\in \mathbb{C}).
\end{align*}
There is the Euler product formula
\begin{align*}
\zeta(s) = \sum_{n=1}^{\infty}{1\over n^s}=
\prod_{p:\,\mathrm{prime}}\big(1-{1\over p^s}\big)^{-1},\quad  (s>1 \, \mathrm{for}\, s\in \mathbb{R}; \Re s >1\, \mathrm{for}\, s\in \mathbb{C}).
\end{align*}
The equality of the series and the product for $\zeta(s)$ is just an analytic way of expressing the fundamental theorem of arithmetic. This is briefly explained
on pp. 15-16 of Baker's book, or in some more detail \S 8.2 of NZM. Taking logarithms in the product formula, we have (subject to the specified restriction
on the domain of $s$)
\begin{align*}
\log \zeta(s) = \log\Big(\prod_{p:\,\mathrm{prime}}\big(1-{1\over p^s}\big)^{-1}\Big) = -\sum_{p:\,\mathrm{prime}}\log\big(1-{1\over p^s}\big).
\end{align*}
Now taking derivatives of both sides, we have
\begin{align*}
{\zeta'(s)\over \zeta(s)} = - \sum_{p:\,\mathrm{prime}}{1\over 1-{1\over p^s}}{\log p\over p^s} = - \sum_{p:\,\mathrm{prime}}(\log p) \Big(
{1\over p^s} +{1\over p^{2s}} + {1\over p^{3s}} + \cdots \Big) = -\sum_{n=1}^{\infty}{\Lambda(n)\over n^s}.
\end{align*}
We also have, from term-by-term differentiation of the series for $\zeta(s)$ that
\begin{align*}
\zeta'(s) = -\sum_{n=1}{\log n\over n^s}.
\end{align*} 
Multiplication of the series for $\displaystyle {\zeta'(s)\over \zeta(s)}$ and $\zeta(s)$ should give the series for $\zeta'(s)$. This would reveal
\begin{align*}
\sum_{d\mid n}\Lambda(d) = \log n.
\end{align*}
But we could have obtained this result directly from the definition of $\Lambda(n)$. Let $n = p_{1}^{a_1}\cdots p_{k}^{a_k}$ (here the $p_i$ are primes
and the $a_i$ are positive integers) be the factorization of $n$ into primes. Then, since $\Lambda(d)$ is non-zero only when $n$ is a prime power, we have
\begin{align*}
\sum_{d\mid n}\Lambda(d) = a_{1}\log p_1 + \cdots a_{k}\log p_k = \log n.
\end{align*}
\vskip .3cm


{\bf 5)} (NZM 8.3.19-20) Show that $\displaystyle \Phi(x) := \sum_{1 \leq n\leq x}\phi(n) ={3\over \pi^2}x^2 + O(x\log x)$ for $x \geq 2$. Show that
the number of pairs of integers $m\leq x, n\leq x,\, (m,n) =1$ is $2\Phi(x)+1$. Deduce that if two integers are chosen at random from the interval
$[1,x]$ then the probability that they are relatively prime is approximately $\displaystyle {6\over \pi^2}$ if $x$ is large.

These are all solved on p.14 of Baker's book. Here we make a direct calculation.
\begin{align*}
\sum_{m,n \leq x \atop (m,n)=1}1 & = \sum_{m,n \leq x}\,\sum_{d\mid m,\, d\mid n}\mu(d) 
= \sum_{d\leq x}\mu(d) \, \sum_{m \leq {x\over d},\, n\leq {x\over d}} 1
= \sum_{d\leq x}\mu(d)\Big({\lfloor x\rfloor\over d}\Big)^2 = \sum_{d\leq x}\mu(d)\Big(\big({x\over d}\big)^2 + O({x\over d})\Big) \\
& = x^2 \sum_{d\leq x}{\mu(d)\over d^2} + O\big(x\sum_{d\leq x}{1\over d}\big) = x^2 \Big( \sum_{d=1}^{\infty}{\mu(d)\over d^2} -
\sum_{d > x}{\mu(d)\over d^2} \Big) +O(x\log x) = {x^2\over \zeta(2)} + O(x\log x) \\
& = {6x^2\over \pi^2} + O(x\log x).
\end{align*}
Here we have used $\displaystyle \sum_{n=1}^{\infty}{\mu(n)\over n^s} = {1\over \zeta(s)}$ for $s>1$ (or for complex $s$, $\Re s > 1$) (Baker  p.16 or
Theorem 8.15 of NZM);
$\displaystyle \zeta(2) =\sum_{n=1}^{\infty}{1\over n^2} ={\pi^2\over 6}$ (appendix A.3 of NZM on pp. 490-491); $\displaystyle \big|\sum_{d > x}{\mu(d)\over d^2}\big| \leq  
\sum_{d > x}{1\over d^2} =O\big({1\over x}\big)$ (the first inequality is obvious, and the $O$-estimate follows from comparing the series 
with the integral $\displaystyle \int_{x}^{\infty}{dx\over x^2}$); $\displaystyle \sum_{d\leq x}{1\over d} = O(\log x)$ (by comparing the series with $\displaystyle
\int_{1}^{x}{du\over u}$).

In fact, for the last result one can easily have the much more precise estimate 
$\displaystyle \sum_{d\leq x}{1\over d} = \log x + \gamma +O\big({1\over x}\big)$. This is done Lemma 8.27 of NZM (pp. 391-392). A recommended reading
on this matter is \S 22.5 of Hardy \& Wright's classic book.



\end{document}


